\documentclass{beamer}
\mode<presentation>
\usepackage{amsmath}
\usepackage{amssymb}
%\usepackage{advdate}
\usepackage{adjustbox}
\usepackage{subcaption}
\usepackage{enumitem}
\usepackage{multicol}
\usepackage{listings}
\usepackage{url}
\def\UrlBreaks{\do\/\do-}
\usepackage{listings}
\usepackage{url}
\def\UrlBreaks{\do\/\do-}
\usetheme{Warsaw}
\usecolortheme{whale}
\setbeamertemplate{footline}
{
  \leavevmode%
  \hbox{%
  \begin{beamercolorbox}[wd=\paperwidth,ht=2.25ex,dp=1ex,right]{author in head/foot}%
    \insertframenumber{} / \inserttotalframenumber\hspace*{2ex} 
  \end{beamercolorbox}}%
  \vskip0pt%
}
\setbeamertemplate{navigation symbols}{}

\providecommand{\nCr}[2]{\,^{#1}C_{#2}} % nCr
\providecommand{\nPr}[2]{\,^{#1}P_{#2}} % nPr
\providecommand{\mbf}{\mathbf}
\providecommand{\pr}[1]{\ensuremath{\Pr\left(#1\right)}}
\providecommand{\qfunc}[1]{\ensuremath{Q\left(#1\right)}}
\providecommand{\sbrak}[1]{\ensuremath{{}\left[#1\right]}}
\providecommand{\lsbrak}[1]{\ensuremath{{}\left[#1\right.}}
\providecommand{\rsbrak}[1]{\ensuremath{{}\left.#1\right]}}
\providecommand{\brak}[1]{\ensuremath{\left(#1\right)}}
\providecommand{\lbrak}[1]{\ensuremath{\left(#1\right.}}
\providecommand{\rbrak}[1]{\ensuremath{\left.#1\right)}}
\providecommand{\cbrak}[1]{\ensuremath{\left\{#1\right\}}}
\providecommand{\lcbrak}[1]{\ensuremath{\left\{#1\right.}}
\providecommand{\rcbrak}[1]{\ensuremath{\left.#1\right\}}}
\theoremstyle{remark}
\newtheorem{rem}{Remark}
\newcommand{\sgn}{\mathop{\mathrm{sgn}}}
\providecommand{\abs}[1]{{$\left$}\vert#1{$\right$}\vert}
\providecommand{\res}[1]{\Res\displaylimits_{#1}} 
\providecommand{\norm}[1]{\lVert#1\rVert}
\providecommand{\mtx}[1]{\mathbf{#1}}
\providecommand{\mean}[1]{E{$\left$}[ #1 {$\right$}]}
\providecommand{\fourier}{\overset{\mathcal{F}}{ \rightleftharpoons}}
%\providecommand{\hilbert}{\overset{\mathcal{H}}{ \rightleftharpoons}}
\providecommand{\system}{\overset{\mathcal{H}}{ \longleftrightarrow}}
%\newcommand{\solution}[2]{\textbf{Solution:}{#1}}
%\newcommand{\solution}{\noindent \textbf{Solution: }}
\providecommand{\dec}[2]{\ensuremath{\overset{#1}{\underset{#2}{\gtrless}}}}
\newcommand{\myvec}[1]{\ensuremath{\begin{pmatrix}#1\end{pmatrix}}}
\let\vec\mathbf

\lstset{
%language=C,
frame=single, 
breaklines=true,
columns=fullflexible
}

\numberwithin{equation}{section}

\title{EE2101 - Control Systems}
\author{ S.LIKHITA \\ EE19BTECH11032 \\Electrical Engineering\\IIT Hyderabad}


\date{September 7, 2020}
\begin{document}

\begin{frame}
\titlepage
\end{frame}

\begin{frame}

\tableofcontents
\end{frame}
\begin{frame}{Problem}
\section{Problem Statement}
\\ \ \ \\
\\ \ \ \\
Find the transfer function G(s) = {\dfrac{{{\theta}_2}(s)}{T(s)}},
\\
for rotational mechanical system shown in the figure.
\begin{figure}
    \includegraphics[width=0.6\linewidth]{Ques.png}
    \label{fig:my_label}
\end{figure}
\end{frame}
\section{Solution}
\subsection{Rotational Mechanical Systems}
\begin{frame}
\frametitle{Rotational Mechanical Systems}
{\textbf{Rotational mechanical systems}} move about a fixed axis. These systems mainly consist of three basic elements. They are moment of inertia, torsional spring and a dashpot.
\\$\bullet$ If a torque is applied to a rotational mechanical system, then it is opposed by opposing torques due to moment of inertia, elasticity and friction of the system. 
\\ $\bullet$ Since the applied torque and the opposing torques are in opposite directions, \textbf{the algebraic sum of torques acting on the system is zero.}

\end{frame}
\begin{frame}
\\ We know that, for:
\\$\bullet${\textbf{Spring:}} \\ 
Torque,\ T(t) = K{$\theta$}(t)
\begin{equation}
  Impedance, T(s)/{\theta}(s) = K
\end{equation} 
\\$\bullet${\textbf{Moment of Inertia:}} \\
Torque,\ T(t) = J(\dfrac{d\omega}{dt})=  J(\dfrac{d^2\theta}{dt^2})\\
\begin{equation}
  Impedance, T(s)/{\theta}(s) = Js^2
\end{equation} 
\end{frame}
\begin{frame}
\\$\bullet${\textbf{Dashpot:}} 
\\a)One free end (i.e if one end is fixed to a reference):\\
T(t) = D{\omega}(t)=  D(\dfrac{d\theta}{dt})\\
\\b)For two free ends,
\\  T(t) = D(\dfrac{d({{\theta}_1}-{\theta}_2)}{dt})\\ 

\\And,
\begin{equation}
   Impedance, T(s)/{\theta}(s) = Ds
\end{equation} 
\end{frame}

\subsubsection{Gears}
\begin{frame}{Gears}
\\  We will assume that connected gears fit perfectly together.
\\ Consider input gear has radius {$r_1$} and is rotated by {$\theta_1$}; output gear has radius {$r_2$}, then {$\theta_2$}=({$r_1$}/{$r_2$}){\theta_1}=({$N_1$}/{$N_2$}){\theta_1}.
\\
where,N1 and N2 are no.of teeth of gears respectively  and radius is proportional to no.of teeth.\\
No.of teeth is inversely proportional to torque. 
\\ \ \\
Assuming there is no energy loss, we have {$T\theta$}= constant..i.e.\\ {$T_1$}{$\theta_1$} = {$T_2$}{$\theta_2$}.
\\ \ \ \\And,{$T_1$}{N_2} = {$T_2$}{N_1}
\end{frame}
\begin{frame}{Solving the problem}
To find the relation between T1 and {$\theta_1$}:
\begin{figure}
    \includegraphics[width=0.5\linewidth]{sol1.png}
    \label{fig:my_label}
\end{figure} 
\\ Consider the torque due to the damper on shaft2:{T_{D2}}=Ds{$\theta_2$}(s)
\\ \ \ {T_{D2}}=Ds{\dfrac{N_1}{N_2}}{\theta_1}(s)
\\And,\\
{T_{D1}}=({$N_1$}/{$N_2$}){T_{D2}}
\\ {T_{D1}}={\dfrac{N_1}{N_2}}[Ds{\dfrac{N_1}{N_2}}{\theta_1}(s)] = {({\dfrac{N_1}{N_2}})^2} Ds {\theta_1}(s)

\end{frame}
\subsection{Problem solving}
\begin{frame}
\\ \ \ \\
 \textbf{Reflecting Impedances to {{$\theta$}_2} \text{ on the system:}}
 \begin{figure}
    \includegraphics[width=0.75\linewidth]{Ques.png}
    \label{fig:my_label}
\end{figure}
\\Due to springs \implies 250 + 3{(50/5)^2}.
\\Due to dashpot \implies [1000({\dfrac{(5)(50)}{(25)(5)}})^2]s.
\end{frame}
\begin{frame}

\\ \ \ \\Due to moment of inertia \implies [200+3{(50/5)^2}+200({\dfrac{(5)(50)}{(25)(5)}})^2]{s}^2.
\\ \ \ \\
Adding up all the effects,\\

 [[250 + 3{(50/5)^2}]+[1000[({\dfrac{(5)(50)}{(25)(5)}})^2]s+[200+3{(50/5)^2}+200({\dfrac{(5)(50)}{(25)(5)}})^2{s}^2]  \ \  {{\theta}_2}(s)=
\begin{equation}
      (50/5)[T(s)] \ \ \ \ \ \
\end{equation}
\\ \ \ \\
\implies [(550)+(4000)s+(500+800){s^2}]{({{\theta}_2}(s))}= 10[T(s)]
\\
 \implies \dfrac{{({{\theta}_2}(s))}}{T(s)}=\dfrac{10}{1300s^2+4000s+550}
\\ \ \ \\
Therefore, the required transfer function G(s) is 
\\ 
\dfrac{{{{\theta}_2}(s)}}{T(s)}=\dfrac{1}{130s^2+400s+55}.
\end{frame}
%\begin{frame}
%\frametitle{Introduction}
%\framesubtitle{Literature}
%%\begin{figure}[t!]
%%    \centering
%%    \begin{subfigure}[t]{0.4\columnwidth}
%%        \centering
%%        \includegraphics[width=\columnwidth]{point_source}
%%        \caption{Single point source}
%%\label{fig3:subfig1}        
%%    \end{subfigure}%
%%    ~ 
%%    \begin{subfigure}[t]{0.4\columnwidth}
%%        \centering
%%        \includegraphics[width=\columnwidth]{pointNoPowerDist_new}
%%        \caption{SNR profile}
%%\label{fig3:subfig2}
%%    \end{subfigure}
%%  %  \caption{Average SNR for a BPP. $N=16$}
%%    \label{fig3}
%%  \end{figure}
%
%\end{frame}
%  
%
%
%%

\end{document}
